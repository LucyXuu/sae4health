% Options for packages loaded elsewhere
\PassOptionsToPackage{unicode}{hyperref}
\PassOptionsToPackage{hyphens}{url}
%
\documentclass[
]{article}
\usepackage{amsmath,amssymb}
\usepackage{iftex}
\ifPDFTeX
  \usepackage[T1]{fontenc}
  \usepackage[utf8]{inputenc}
  \usepackage{textcomp} % provide euro and other symbols
\else % if luatex or xetex
  \usepackage{unicode-math} % this also loads fontspec
  \defaultfontfeatures{Scale=MatchLowercase}
  \defaultfontfeatures[\rmfamily]{Ligatures=TeX,Scale=1}
\fi
\usepackage{lmodern}
\ifPDFTeX\else
  % xetex/luatex font selection
\fi
% Use upquote if available, for straight quotes in verbatim environments
\IfFileExists{upquote.sty}{\usepackage{upquote}}{}
\IfFileExists{microtype.sty}{% use microtype if available
  \usepackage[]{microtype}
  \UseMicrotypeSet[protrusion]{basicmath} % disable protrusion for tt fonts
}{}
\makeatletter
\@ifundefined{KOMAClassName}{% if non-KOMA class
  \IfFileExists{parskip.sty}{%
    \usepackage{parskip}
  }{% else
    \setlength{\parindent}{0pt}
    \setlength{\parskip}{6pt plus 2pt minus 1pt}}
}{% if KOMA class
  \KOMAoptions{parskip=half}}
\makeatother
\usepackage{xcolor}
\usepackage[margin=1in]{geometry}
\usepackage{color}
\usepackage{fancyvrb}
\newcommand{\VerbBar}{|}
\newcommand{\VERB}{\Verb[commandchars=\\\{\}]}
\DefineVerbatimEnvironment{Highlighting}{Verbatim}{commandchars=\\\{\}}
% Add ',fontsize=\small' for more characters per line
\usepackage{framed}
\definecolor{shadecolor}{RGB}{248,248,248}
\newenvironment{Shaded}{\begin{snugshade}}{\end{snugshade}}
\newcommand{\AlertTok}[1]{\textcolor[rgb]{0.94,0.16,0.16}{#1}}
\newcommand{\AnnotationTok}[1]{\textcolor[rgb]{0.56,0.35,0.01}{\textbf{\textit{#1}}}}
\newcommand{\AttributeTok}[1]{\textcolor[rgb]{0.13,0.29,0.53}{#1}}
\newcommand{\BaseNTok}[1]{\textcolor[rgb]{0.00,0.00,0.81}{#1}}
\newcommand{\BuiltInTok}[1]{#1}
\newcommand{\CharTok}[1]{\textcolor[rgb]{0.31,0.60,0.02}{#1}}
\newcommand{\CommentTok}[1]{\textcolor[rgb]{0.56,0.35,0.01}{\textit{#1}}}
\newcommand{\CommentVarTok}[1]{\textcolor[rgb]{0.56,0.35,0.01}{\textbf{\textit{#1}}}}
\newcommand{\ConstantTok}[1]{\textcolor[rgb]{0.56,0.35,0.01}{#1}}
\newcommand{\ControlFlowTok}[1]{\textcolor[rgb]{0.13,0.29,0.53}{\textbf{#1}}}
\newcommand{\DataTypeTok}[1]{\textcolor[rgb]{0.13,0.29,0.53}{#1}}
\newcommand{\DecValTok}[1]{\textcolor[rgb]{0.00,0.00,0.81}{#1}}
\newcommand{\DocumentationTok}[1]{\textcolor[rgb]{0.56,0.35,0.01}{\textbf{\textit{#1}}}}
\newcommand{\ErrorTok}[1]{\textcolor[rgb]{0.64,0.00,0.00}{\textbf{#1}}}
\newcommand{\ExtensionTok}[1]{#1}
\newcommand{\FloatTok}[1]{\textcolor[rgb]{0.00,0.00,0.81}{#1}}
\newcommand{\FunctionTok}[1]{\textcolor[rgb]{0.13,0.29,0.53}{\textbf{#1}}}
\newcommand{\ImportTok}[1]{#1}
\newcommand{\InformationTok}[1]{\textcolor[rgb]{0.56,0.35,0.01}{\textbf{\textit{#1}}}}
\newcommand{\KeywordTok}[1]{\textcolor[rgb]{0.13,0.29,0.53}{\textbf{#1}}}
\newcommand{\NormalTok}[1]{#1}
\newcommand{\OperatorTok}[1]{\textcolor[rgb]{0.81,0.36,0.00}{\textbf{#1}}}
\newcommand{\OtherTok}[1]{\textcolor[rgb]{0.56,0.35,0.01}{#1}}
\newcommand{\PreprocessorTok}[1]{\textcolor[rgb]{0.56,0.35,0.01}{\textit{#1}}}
\newcommand{\RegionMarkerTok}[1]{#1}
\newcommand{\SpecialCharTok}[1]{\textcolor[rgb]{0.81,0.36,0.00}{\textbf{#1}}}
\newcommand{\SpecialStringTok}[1]{\textcolor[rgb]{0.31,0.60,0.02}{#1}}
\newcommand{\StringTok}[1]{\textcolor[rgb]{0.31,0.60,0.02}{#1}}
\newcommand{\VariableTok}[1]{\textcolor[rgb]{0.00,0.00,0.00}{#1}}
\newcommand{\VerbatimStringTok}[1]{\textcolor[rgb]{0.31,0.60,0.02}{#1}}
\newcommand{\WarningTok}[1]{\textcolor[rgb]{0.56,0.35,0.01}{\textbf{\textit{#1}}}}
\usepackage{graphicx}
\makeatletter
\newsavebox\pandoc@box
\newcommand*\pandocbounded[1]{% scales image to fit in text height/width
  \sbox\pandoc@box{#1}%
  \Gscale@div\@tempa{\textheight}{\dimexpr\ht\pandoc@box+\dp\pandoc@box\relax}%
  \Gscale@div\@tempb{\linewidth}{\wd\pandoc@box}%
  \ifdim\@tempb\p@<\@tempa\p@\let\@tempa\@tempb\fi% select the smaller of both
  \ifdim\@tempa\p@<\p@\scalebox{\@tempa}{\usebox\pandoc@box}%
  \else\usebox{\pandoc@box}%
  \fi%
}
% Set default figure placement to htbp
\def\fps@figure{htbp}
\makeatother
\setlength{\emergencystretch}{3em} % prevent overfull lines
\providecommand{\tightlist}{%
  \setlength{\itemsep}{0pt}\setlength{\parskip}{0pt}}
\setcounter{secnumdepth}{-\maxdimen} % remove section numbering
\usepackage{bookmark}
\IfFileExists{xurl.sty}{\usepackage{xurl}}{} % add URL line breaks if available
\urlstyle{same}
\hypersetup{
  pdftitle={deploy\_sae4health},
  hidelinks,
  pdfcreator={LaTeX via pandoc}}

\title{deploy\_sae4health}
\author{}
\date{\vspace{-2.5em}2025-11-06}

\begin{document}
\maketitle

In this vignette, we will describe how to deploy sae4health both locally
and to the UW stat server.

\subsection{Fork and clone the SAE4Health
package}\label{fork-and-clone-the-sae4health-package}

Frok the most up to date repo of SAE4Health from
\url{https://github.com/wu-thomas/sae4health} and clone the repo to
local computer

\subsection{Rtools}\label{rtools}

Download competible verion of rtools from
\url{https://cran.r-project.org/bin/windows/Rtools/} if not already done
so.

\subsection{Prepare renv}\label{prepare-renv}

This shiny app uses renv to manage package and versions. Run the
following code to restore packages from renv, with xtra steps are needed
to install INLA. renv::retore() will be take a while to restore all
packages.

\begin{Shaded}
\begin{Highlighting}[]
\FunctionTok{install.packages}\NormalTok{(}\StringTok{"renv"}\NormalTok{)}
\NormalTok{renv}\SpecialCharTok{::}\FunctionTok{activate}\NormalTok{()}
\FunctionTok{options}\NormalTok{(}\AttributeTok{repos =} \FunctionTok{c}\NormalTok{(}\FunctionTok{getOption}\NormalTok{(}\StringTok{"repos"}\NormalTok{), }\AttributeTok{INLA =} \StringTok{"https://inla.r{-}inla{-}download.org/R/testing"}\NormalTok{))}
\NormalTok{renv}\SpecialCharTok{::}\FunctionTok{install}\NormalTok{(}\StringTok{"INLA"}\NormalTok{)}
\FunctionTok{options}\NormalTok{(}\AttributeTok{repos =} \FunctionTok{c}\NormalTok{(}\AttributeTok{CRAN =} \StringTok{"https://cloud.r{-}project.org"}\NormalTok{)) }\CommentTok{\# change back}
\NormalTok{renv}\SpecialCharTok{::}\FunctionTok{update}\NormalTok{(}\StringTok{"INLA"}\NormalTok{)}
\NormalTok{renv}\SpecialCharTok{::}\FunctionTok{restore}\NormalTok{()}
\end{Highlighting}
\end{Shaded}

\subsection{setup local server and run
app}\label{setup-local-server-and-run-app}

Some data, such as county recode data and other country meta data, are
saveing in an online server for the app to fetch from. Setup a local
server and store a part of the data to locally test functionality and
connection to server.

First, create a folder on local desk top, for example, at
C:/Users/username/Desktop/sae4healthServer.

Copy and run the following code in \emph{Terminal} to set up a local
server:

cd C:/Users/username/Desktop/sae4healthServer py -3 -m http.server 8000
\# run this line for windows python3 -m http.server 8000 \# run this
line for mac

now, a message should appear in terminal with something like ``Serving
HTTP on :: port 8000 (\url{http://}{[}::{]}:8000/) \ldots{}'',
indicating the local server deployment is successful.

To run and test app on the local server, run the following code in
console. The app will be ran in a separate window, deployed as server
version that connects to the local server set up.

\begin{Shaded}
\begin{Highlighting}[]
\NormalTok{pkgload}\SpecialCharTok{::}\FunctionTok{load\_all}\NormalTok{(}\AttributeTok{export\_all =} \ConstantTok{FALSE}\NormalTok{,}\AttributeTok{helpers =} \ConstantTok{FALSE}\NormalTok{,}\AttributeTok{attach\_testthat =} \ConstantTok{FALSE}\NormalTok{)}
\FunctionTok{options}\NormalTok{( }\StringTok{"golem.app.prod"} \OtherTok{=} \ConstantTok{TRUE}\NormalTok{)}
\CommentTok{\#test on local server}
\NormalTok{sae4health}\SpecialCharTok{::}\FunctionTok{run\_app}\NormalTok{(}\AttributeTok{server\_link=}\StringTok{\textquotesingle{}http://localhost:8000/\textquotesingle{}}\NormalTok{)}
\end{Highlighting}
\end{Shaded}

To imitate the folder structure on server, store the meta data and DHS
estimate and any country specific recode data as the structure below:

sae4healthServer/ └── DHS\_survey\_dat/ ├── DHS\_meta\_data/ │ ├──
DHS\_api\_est.rda \emph{ │ ├── DHS\_meta\_preload.rda } └── NG/ * │ ├──
DHS 2024 \textbar{} \textbar{} ├── {[}recode name 1{]}.rds \textbar{}
\textbar{} ├── {[}recode name 2{]}.rds \textbar{} \textbar{} ├── {[}geo
info{]}.rds \textbar{} \textbar{} ├── \ldots{}

\begin{itemize}
\tightlist
\item
  represents files needed for app to run
\end{itemize}

Where NG can be any country with its recode and GPS data with structure.

All server data is stored in dropbox, ask Jon for access if needed. It
will also be helpful to see the dropbox folder structure of data to know
where to place the folders and files

In this example, Nigeria 2024 can be ran on the locally deployed version
of app for testing purpose. More country recode can be added if testing
is needed.

\subsection{deploy to stats server}\label{deploy-to-stats-server}

\subsubsection{For newly deployed app}\label{for-newly-deployed-app}

After gaining admin access to the domain, enter through
\url{https://rsc.stat.washington.edu/connect/\#/apps/60/access}, click
info on the right tab, and copy the appId.

Run the following code to login and deploy to connect to the existing
domain:

\begin{Shaded}
\begin{Highlighting}[]
\NormalTok{rsconnect}\SpecialCharTok{::}\FunctionTok{deployApp}\NormalTok{(}
  \AttributeTok{appDir =} \StringTok{"."}\NormalTok{,}
  \AttributeTok{account =} \StringTok{"[your netID]"}\NormalTok{,}
  \AttributeTok{server =} \StringTok{"rsc.stat.washington.edu"}\NormalTok{,}
  \AttributeTok{appId =} \StringTok{"[copied appId]"}
\NormalTok{)}
\end{Highlighting}
\end{Shaded}

This command deploys the app to Posit Connect using the startup code
defined in app.R. The newly deployed version will appear on
\url{https://rsc.stat.washington.edu/sae4health/} and can be managed on
\url{https://rsc.stat.washington.edu/connect/\#/apps/60/access}.

\subsubsection{Future deployment}\label{future-deployment}

Running the previous code works fine in future deployment. One other
valid method is to first run the following run\_app() function, and then
click the republish button on the right corner of the newly opened
window. This helps you check functionality in the local window before
deployment.

\begin{Shaded}
\begin{Highlighting}[]
\NormalTok{pkgload}\SpecialCharTok{::}\FunctionTok{load\_all}\NormalTok{(}\AttributeTok{export\_all =} \ConstantTok{FALSE}\NormalTok{,}\AttributeTok{helpers =} \ConstantTok{FALSE}\NormalTok{,}\AttributeTok{attach\_testthat =} \ConstantTok{FALSE}\NormalTok{)}
\FunctionTok{options}\NormalTok{( }\StringTok{"golem.app.prod"} \OtherTok{=} \ConstantTok{TRUE}\NormalTok{)}
\NormalTok{sae4health}\SpecialCharTok{::}\FunctionTok{run\_app}\NormalTok{(}\AttributeTok{server\_link=}\StringTok{\textquotesingle{}https://sites.stat.washington.edu/sae4health/\textquotesingle{}}\NormalTok{)}
\end{Highlighting}
\end{Shaded}


\end{document}
